% !TEX root = ../main.tex

\chapter{正文文字格式}

\section{论文正文}
论文正文是主体,一般由标题、文字叙述、图、表格和公式等部分构成 \cite{Yang1999}。一般可包括理论分析、计算方法、实验装置和测试方法,经过整理加工的实验结果分析和讨论,与理论计算结果的比较以及本研究方法与已有研究方法的比较等,因学科性质不同可有所变化。
\par 论文内容一般应由十个主要部分组成,依次为:1.封面,2.中文摘要,3.英文摘要,4.目录,5.符号说明,6.论文正文,7.参考文献,8.附录,9.致谢,10.攻读学位期间发表的学术论文目录\cite{Yu2012}。
\par 以上各部分独立为一部分,每部分应从新的一页开始,且纸质论文应装订在论文的右侧。


\section{字数要求}
\subsection{本科论文要求}
各学科和学院自定。理工科研究类论文一般不少于2万字,设计类一般不少于1.5万字;医科、文科类论文一般不少于1万字。

\section{本章小结}
本章介绍了……

